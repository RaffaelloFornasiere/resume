\documentclass[a4paper,9pt]{article}
\usepackage[left=0.4in, right=0.4in, top=0.4in, bottom=0.5in]{geometry}
\usepackage{hyperref}
\usepackage{amssymb}
% Define a reusable section template
\newcommand{\cvsection}[2]{
    \vspace{0.3cm} % Add spacing before each section
    \noindent {\large \textbf{#1}} % Section Title
    \begin{itemize}
        \renewcommand{\labelitemi}{}
        \small #2 % Bullet Points
    \end{itemize}
}
\usepackage{enumitem}
\setlist[itemize,1]{left=-10pt} % Remove extra indentation for first-level items
\setlist[itemize,2]{left=0.8em} % Slightly indent second-level items
\setlist[itemize]{left=0pt, topsep=5pt, itemsep=2pt, parsep=0pt}



\begin{document}

% Header
\noindent{\LARGE \textbf{Raffaello Fornasiere}}

\vspace{0.2cm}
\noindent
\begin{minipage}{\linewidth}
    \begin{center}
        Email: \href{mailto:fornasiere.raffaello@gmail.com}{fornasiere.raffaello@gmail.com} \hfill
        Linkedin: \href{https://linkedin.com/in/raffaello-fornasiere}{raffaello-fornasiere} \hfill
        Website: \href{https://rf-98.com}{rf-98.com}\hfill
        Github: \href{https://github.com/RaffaelloFornasiere}{RaffaelloFornasiere}\hfill
    \end{center}
\end{minipage}

\cvsection{Summary}{
    \item I am a software engineer with practical experience in LLM integration and AI agent development. I have built AI-powered applications in healthcare and research contexts, working across the full stack. My background combines AI/ML knowledge with hands-on development skills to deliver production-ready solutions.
}

\cvsection{Professional Experience}{
    \item \textbf{AI Agent Developer}, Freelance for William Saunders, Alignment Science Researcher at Anthropic (May 2025 - Present)
    \begin{itemize}
        \item Building an AI conversational assistant, with manageable memory and personalizable with many tools and external services.
        \item The agent has a web interface with chat, text-to-speech and speech-to-speech interfaces.
        \item Tech stack: FastAPI, React, Claude API, RAG, event-driven architecture, and various third-party APIs.
    \end{itemize}
    \item \textbf{Lead Software Engineer - Front-End}, Infiniteloop (October 2022 - Present)
    \begin{itemize}
        \item Prototyped AI healthcare applications using LLMs, exploring feasibility for medical report automation (related to my thesis work).
        \item Designed, built and maintained user interfaces using Angular, ensuring seamless integration with JHipster framework.
        \item Developed reusable JHipster framework extensions (e.g. an extended JHipster Criteria API, a custom Spring Data JPA repository for bidirectional relationship synchronization).
        \item Lead the development of different products. 
        \item Mentored two junior developers and promoted best practices to improve code quality.
    \end{itemize}
    \item \textbf{Junior Full Stack Developer}, Prodigys Group (July 2021 - September 2022)
    \begin{itemize}
        \item Developed and maintained both front-end and back-end components using Angular, Spring Boot, and Django.
        \item Optimized database performance through SQL query refactors, reducing execution times from hours to minutes.
    \end{itemize}
}

\cvsection{Education}{
    \item \textbf{Master's in Computer Science and Engineering}, Politecnico di Milano (Graduated July 2024)
    \begin{itemize}
        \item Thesis: Exploring the Potential of Lightweight LLMs for Medication and Timeline Extraction.
        \item Developed a web application integrating LLMs for medical information extraction, \textbf{tested in three Italian hospitals}.
        \item Published research: ``Medical Information Extraction with Large Language Models'' (ACL 2024).
    \end{itemize}
    \item \textbf{Bachelor's in Electronics Engineering}, Università degli Studi di Udine (Graduated July 2020)
    \begin{itemize}
        \item Thesis: Optimization of Digital Circuit Propagation Times Using Genetic Algorithms.
    \end{itemize}
}

\cvsection{Technical Skills}{
    \item \textbf{AI/ML \& LLM Integration:}
    \begin{itemize}
        \item Production experience with LLM APIs (OpenAI, Anthropic Claude), and prompt engineering.
        \item Experience with local/lightweight models (with llamacpp) for privacy-sensitive applications.
        \item Agent development: Autonomous systems, tool integrations, streaming responses and conversational interfaces.
        \item NLP: Text extraction, information retrieval, medical language processing.
        \item Libraries: LangChain, llamacpp, Hugging Face Transformers, TensorFlow.
        \item Academic background in neural networks (CNNs, RNNs, transformers), optimization methods.
    \end{itemize}
    \item \textbf{Languages:}
    \begin{itemize}
        \item Primary: TypeScript/JavaScript, Python, Java.
        \item Also experienced with: C++, SQL.
        \item Others: CSS, HTML. 
    \end{itemize}
    \item \textbf{Frameworks, Libraries and Tools:}
    \begin{itemize}
        \item Back-end: FastAPI, Django, Spring Boot, JHipster.
        \item Front-end: React, Vue.js, Angular.
    \item Others: Docker, Git, Tailwind CSS, Streamlit, Spring Data JPA, Postgres/PostgreSQL, SQLite.
    \end{itemize}
}


\cvsection{Publications}{
    \item \textbf{Medical Information Extraction with Large Language Models} \\
    \textit{Published at ICNLSP-2024, ACL Anthology (A* ranked by CORE, ~20-25\% acceptance rate)} \\
    \url{https://aclanthology.org/2024.icnlsp-1.47/}
    \begin{itemize}
        \item Authors: \textbf{Raffaello Fornasiere}, Nicolò Brunello, Vincenzo Scotti, Mark Carman.
    \end{itemize}
}
\pagebreak


\cvsection{Academic Projects}{
    \item \textbf{Medical Image-Caption Matching with Transformers}
    \begin{itemize}
       \item Reimplemented CLIP model from scratch for medical radiology applications using ResNet vision encoder and custom transformer text encoder.
       \item Trained and fine-tuned both models using contrastive learning, pre-trained BERT integration, caption preprocessing, and medical concept augmentation techniques.
       \item Developed evaluation methods for assessing image-caption alignment in medical domain.
    \end{itemize}
    \item \textbf{Online Learning for Dynamic Pricing in E-commerce}
    \begin{itemize}
        \item Exploration work implementing multi-armed bandit algorithms (GP-TS, GP-UCB) with social influence modeling to optimize advertising budget allocation across product campaigns.
    \end{itemize}
    \item \textbf{Deep Learning Projects with CNNs}
    \begin{itemize}
        \item \textbf{Species Classification:} Implemented transfer learning with VGG16/19 and EfficientNet models, applying data augmentation techniques (CutMix, MixUp) and ensemble methods to achieve 86.91\% accuracy on image classification tasks.
        
        \item \textbf{Time Series Classification:} Developed CNN-LSTM hybrid architectures for sequential data analysis, experimenting with preprocessing techniques including normalization and noise augmentation to reach 70\% classification accuracy.
    \end{itemize}
}

\cvsection{Other Experiences}{
    \item \textbf{ML4Good AI Safety Bootcamp}
    \begin{itemize}
        \item Completed EPFL's iteration of the \href{https://www.ml4good.org/}{ML4Good} AI Safety and ML Bootcamp (February 2023).
    \end{itemize}
}

\cvsection{Community Involvement}{
    \item \textbf{Volunteer developer for hometown community}
    \begin{itemize}
        \item Built website to publish local competition status with real-time updates through Telegram bot.
        \item Developed web application to emulate Italian TV game show, managing multiplayer gameplay across mobile devices.
    \end{itemize}
}

\end{document}

